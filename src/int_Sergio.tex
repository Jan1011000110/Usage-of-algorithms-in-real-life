\section{Entrevista 2:}

\textbf{Primerament, podries introduir-te breument?} \newline

Soc el Sergio Domínguez, estudiant de 2n de batxillerat. Vaig començar amb la programació competitiva el juliol del 2021 amb el curs d'estiu de la UPC. Per practicar he utilitzat codeforces principalment, encara que també he assistit al programa de Leagues of Code.
Les meves principals fites han estat: Candidate Master a Codeforces, 2n a l'OIcat (Or), 4t a l'OIE (Or), 175 a la IOI (Bronze). \newline


Per a mi els algorismes de programació competitiva no tenen gaires usos pràctics a la vida real encara que sí que he arribat a usar alguns d'ells en algunes ocasions. Un exemple és que per calcular potències o multiplicacions de cap més ràpid es pot emprar una idea semblant a la de \emph{binary exponentiation}.
També cal dir que encara que no siguin algorismes de programació competitiva, el fet de practicar programació competitiva m'ha fet millorar molt en àrees adjacents, com per exemple matemàtiques. L'indicador més clar són les proves cangur. A les del curs 2020-2021 vaig quedar 825 de 10000, en el millor 8 per cent, mentre que a les del curs 2021-2022 (després d'haver començat programació competitiva) vaig quedar 31 de 4500, en el millor 0,7 per cent. \newline

\textbf{Creus que les persones que no tenen coneixements sobre els algorismes emprats en la programació competitiva, a vegades els usen inconscientment? Si és així, pots donar algun exemple?} \newline

No crec que sigui molt comú que gent que no conegui algorismes de programació competitiva els faci servir a la vida quotidiana, encara que potser fent matemàtiques o programació normal pots arribar a fer servir algun algorisme. Per exemple, per calcular l'enèsim nombre de Fibonacci (sense la fórmula), la manera més senzilla és fer ús de la programació dinàmica (dp[i] = dp[i-1]+dp[i-2]), ja que aquesta fórmula és la definició dels nombres de Fibonacci. Aquest algorisme té cost $O(n)$, i encara que està lluny del més eficient (Que és $O(log n)$), et permet calcular nombres de Fibonacci fins a $10^7$ en un segon, que és més del que pots necessitar.\newline

\textbf{Creus que els algorismes emprats en la programació competitiva estan present en la tecnologia? Si és així, pots donar algun exemple?} \newline

Jo crec que encara que la majoria dels algorismes de programació competitiva no estan presents a la vida real ni tenen cap us fora de la programació competitiva, hi ha un sorprenent nombre d'ells que s'utilitzen en tecnologies de les quals depenem cada dia (moltes vegades versions optimitzades més complexes). Un exemple molt bo d'això és l'algorisme $fft$ (Fast Fourier Transform). Aquest algorisme s'usa per fer problemes de programació competitiva, però també s'empra en una gran varietat de situacions. Té aplicacions en telecomunicacions i processament d'àudio, i és essencial per fer funcionar la tecnologia que tenim avui en dia a la velocitat que ho fa. Un altre cas en el qual un algorisme de programació competitiva s'empra a la vida real és per exemple en el GPS. Per decidir la ruta més curta, el GPS fa servir un algorisme que és una modificació de Dijkstra's, que utilitzant heurístiques com A* i precalculant rutes entre punts transitats es torna molt més eficient. \newline

\textbf{En la teva opinió, saber aquests algorismes t'ajuda a entendre millor com funciona la tecnologia? Si és així, explica't breument.} \newline

Jo crec que els algorismes de programació competitiva no tenen massa relació amb el comportament de la gent, ja que aquests algorismes busquen solucions exactes als problemes, mentre que la gent (i el cervell) fa eleccions basades en probabilitat. \newline

\textbf{En la teva opinió, saber aquests algorismes t'ajuda a entendre millor com funciona la tecnologia? Si és així, explica't breument.} \newline

Per a mi una de les parts més interessants de la programació competitiva és que et permet emular i programar fins a cert punt algunes de les tecnologies que utilitzem en el dia a dia. L'exemple més clar és Google Maps, però també hi ha molts altres casos en els quals s'empren aquest tipus d'algorismes (Bases de dades, etc.). \newline

\textbf{Finalment, podries fer una breu conclusió?} \newline

En conclusió, encara que no tots els algorismes de programació competitiva tinguin un ús en la vida real, aquests t'ajuden a entendre la tecnologia que t'envolta, i t'entrena per resoldre tota mena de problemes lògico-matemàtics d'una manera molt divertida.