\chapter{Introducció}

%Actualment, vivim en un món en el qual la programació és un element molt present en les nostres vides, de fet, constantment estem connectats a internet, xarxes socials o mirem una pe\lgem ícula a Netflix, un vídeo a YouTube, etc. \newline%

Actualment, vivim en una societat altament digitalitzada, en la qual la programació és una part molt important de les nostres vides. Les tecnologies que fem servir diàriament utilitzen algorismes, alguns relativament senzills, i d'altres d'una gran complexitat. \newline

Malgrat no sempre tenir-ho present, d'una manera o d'altra estem utilitzant recursos tecnològics digitals basats en algorismes: òbviament quan fem anar el mòbil o l'ordinador, però també en infinitat d'altres situacions diàries com ara en utilitzar el navegador del cotxe, quan fem fotos amb el mòbil (fotografia computacional), quan busquem una pe\lgem ícula en una plataforma d'\emph{streaming}, quan anem en cotxe i triem un mode de conducció (ecològic, normal o esportiu), quan el cotxe que conduïm incorpora alguna tecnologia de conducció autònoma, etc. \newline

Aquestes tecnologies estan pensades per fer-nos la vida més fàcil (tot i que a vegades ens la compliquen!), i són en gran part transparents, invisibles als nostres ulls, gairebé màgiques. \newline 

Amb aquest treball de recerca pretenc ajudar el lector a entendre el seu funcionament, explicant alguns d'aquests algorismes en les quals es basen. Concretament, em centraré en un recull dels algorismes de programació competitiva més populars, provant de donar exemples de la seva aplicació pràctica a la vida real, resolent un problema de programació competitiva relacionat amb aquest i comentant breument les diferències entre la seva aplicació teòrica i l'aplicació pràctica.  \newline


%Es a més, per a cada algorisme, hi afegiré i resoldré un problema de programació competitiva relacionat amb aquest per tal de fer una breu comparació. \newline \newline%

Coneixerem la programació competitiva, que tracta de resoldre problemes algorísmics, lògics i matemàtics de forma eficient i ràpida. \newline

(Tots els codis emprats en el transcurs del treball de recerca estan penjats al meu $Github$ \url{https://github.com/JANPROGRAMER/CODIS-TR} per si els voleu veure, copiar, emprar, etc. Si voleu provar-los amb el fi d’entendre'ls millor, us recomano un compilador en línia:  \url{https://replit.com/})
