\section{Algorismes Greedy A La Vida Real}

La majoria de decisions que prenem en el nostre dia i en la vida en general, segueixen una estratègia $Greedy$, ja que quasi sempre intentem prendre la decisió que més ens beneficia en el moment de prendre aquesta. \newline

Per exemple, a l'hora de completar un examen, una estratègia $Greedy$ podria consistir a resoldre els exercicis en ordre creixent de dificultat (primer els més fàcils i finalment els més difícils).

Un altre exemple, quan anem a comprar qualsevol classe de producte, una estratègia $Greedy$ podria consistir a comprar la marca més econòmica d'aquell producte, ja que així estalviem diners, una altra estratègia $Greedy$ possible seria comprar la marca més cara del producte, puix que així, probablement estem agafant un producte de major qualitat. \newline

En poques paraules, a l'hora de prendre una decisió, tenim innumerables estratègies $Greedy$ possibles a seguir i quasi mai cap és cent per cent correcte.
\newline



Un exemple quotidià de l'ús de l'algorisme $Greedy$, és quan anem a comprar i volem pagar amb el menor nombre de monedes possibles.

Si per exemple hem de pagar 163 cèntims, primer busquem la moneda més gran que tinguem que no sobrepassi els 163 cèntims, si tinguéssim tots els valors de monedes, primer pagaríem 1 euro (100 cèntims), ens restarien 63 cèntims a pagar i de nou busquem la moneda més eficient, en aquest cas seria una moneda de 50 cèntims, ens quedarien 13 cèntims a pagar, altra vegada busquem la moneda més gran que no sobrepassi la xifra de 13 cèntims que seria la moneda de 10 cèntims, ens quedaran tan sols 3 cèntims a pagar que no ens quedaria més remei que pagar-los amb 1 moneda de 2 cèntims i 1 moneda d'1 cèntim.
\newline

En aquest cas l'estratègia $Greedy$ consisteix a agafar la moneda més gran que no sobrepassi els diners restants a pagar. Amb el sistema de monedes a España, aquesta estratègia és l'òptima sempre, però si tinguéssim monedes de tots els cèntims possibles, aquesta estratègia no funcionaria sempre, per exemple:

Si tenim les següents monedes [1,3,4] (cèntims) i hem de pagar 6 cèntims, doncs amb l'estratègia $Greedy$ anterior, agafaríem les següents monedes [4,1,1] però realment l'opció més bona seria agafar les següents monedes [3,3].

Per resoldre aquest problema de manera òptima amb monedes de qualsevol valor, haurem d'utilitzar la \emph{programació dinàmica} (la veurem més endavant). \newpage

En el següent codi, podem observar la implementació de l'estratègia $Greedy$ de la qual hem parlat.
\newline

\begin{lstlisting}
int main(){
    int preu_total;
    cin >> preu_total; // Input -> 279
    int preu_a_pagar = preu_total;
    
    vector<int> monedes = {200,100,50,20,10,5,2,1};
    vector<int> monedes_utilitzades;
    int index = 0;
    int monedes_pagades = 0;
    
    while (preu_a_pagar > 0){
        if (preu_a_pagar >= monedes[index]){
            preu_a_pagar -= monedes[index];
            monedes_utilitzades.push_back(monedes[index]);
            monedes_pagades++;
        } else {
            index++;
        }
    }
    cout << "Hem pagat: " << preu_total << " cèntims amb: "
    << monedes_pagades << " monedes." << endl;
    cout << "{";
    for (auto e : monedes_utilitzades){
        cout << e << ",";
    cout << "}";
    
    // Output -> Introdueix el preu a pagar: Hem pagat: 279 cèntims amb:
    // 6 monedes {200,50,20,5,2,2}
}

\end{lstlisting}