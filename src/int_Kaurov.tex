\section{Entrevista 1:}

Per respondre les preguntes que he preparat, he entrevistat a Innokentiy Kaurov, un jove programador competitiu amb grans coneixements d'algorísmia. \newline

\textbf{Primerament, podries introduir-te breument?} \newline

Em dic Innokentiy Kaurov, vaig començar la programació competitiva el desembre de 2020. Els majors èxits inclouen: 1a posició a l'Olimpíada Informàtica Catalana, 5a posició a l'Olimpíada Informàtica Espanyola, medallista en competicions nacionals Russes, també he assolit el rang de Master a Codeforces i Platí a la USACO. \newline

\textbf{En la teva opinió, els algorismes emprats en la programació competitiva tenen algun ús a la vida real excloent la programació competitiva? Si és així, pots donar algun exemple?} \newline

Realment depèn molt de l'algorisme. Per exemple, els algorismes de grafs com BFS, Dijkstra´s s'empren àmpliament en tecnologies de mapes, per exemple Google Maps, mentre que els algorismes de teoria de nombres s'utilitzen en xifratge.

Tanmateix, en la programació competitiva també existeixen alguns algorismes populars que és completament improbable que s'apliquin al món real: prenem, per exemple, l'arbre de segments. La majoria de les empreses mai no farien una pregunta d'arbre de segment a les seves entrevistes, simplement perquè mai s'empra!

Alguns altres exemples d'algoritmes de la programació competitiva famosos, però pràcticament inútils són: l'algoritme de Mo's, la descomposició Heavy-Light, l'arbre Li-Chao, Convex Hull Trick, etc. Tots aquests algorismes es van crear per accelerar solucions a tipus de problemes molt particulars, que són molt poc probable que apareguin a la vida real.

En conclusió, alguns dels algorismes bàsics de la programació competitiva són molt útils a la vida real, en canvi, la majoria dels algorismes més complexos són totalment inútils fora de la programació competitiva.
\newline

\textbf{Creus que les persones que no tenen coneixements sobre els algorismes emprats en la programació competitiva, a vegades els usen inconscientment? Si és així, pots donar algun exemple?}
\newline

A vegades sí, crec que la gent a vegades fa servir idees sobre algorismes de la programació competitiva a les classes de mates, per exemple si es demana a algú que calculi una potència relativament gran d'un nombre (per exemple, $3^10$) sense una calculadora, acostumen a fer servir algunes tècniques de dividir i vèncer (per exemple, calculant primer $3^5$ i després el seu quadrat), en lloc de multiplicar simplement per 3 deu vegades.

Aquest tipus d'idea s'utilitza àmpliament en la programació competitiva, per exemple en l'exponenciació binària i els algorismes d'elevació binària.

De fet, reflexionem sobre com ensenya l'escola a construir la seqüència de Fibonacci: comença amb [1, 1]; a continuació, a cada pas, agafa els dos últims nombres de la seqüència, suma'ls i afegeix el resultat al final de la seqüència. Sense saber-ho, hem fet servir la programació dinàmica d'ençà que hem après la seqüència de Fibonacci!
\newline

\textbf{Creus que els algorismes emprats en la programació competitiva estan present en la tecnologia? Si és així, pots donar algun exemple?}
\newline

Sí, de fet un dels meus professors va treballar prèviament en una companyia de taxis, i allà es va enfrontar amb el següent problema: l'aplicació de taxis, en els dies de molta feina, era molt lenta a causa del fet que l'algoritme de cerca de taxis tenia una complexitat temporal massa elevada. Va solucionar aquest problema aplicant l'algorisme Hongarès, relativament popular a la programació competitiva.
\newline

\textbf{En la teva opinió, saber aquests algorismes t'ajuda a entendre millor com funciona la tecnologia? Si és així, explica't breument.}
\newline

Tot depèn d'en quina àrea de la tecnologia ens trobem, per exemple els algorismes de grafs m'ajuden relativament a entendre com funciona la recerca de camins a les aplicacions de mapes, però els algorismes utilitzats en les xarxes socials són realment diferents del contingut que estudiem en la programació competitiva, per tant, els algorismes emprats en la programació competitiva no crec que ens ajudin gaire a entendre els algorismes emprats en les xarxes socials, principalment perquè són algorismes heurístics.

A la programació competitiva, realment només estudiem algorismes exactes, no algorismes d'aproximació o heurístics, això és degut al fet que el creador de problemes sempre trobaria un cas que fallaria les dues anteriors.

En el món real, les dades que rep el nostre programa són força aleatòries, i aquest fet és aprofitat pels algorismes per aconseguir un bon resultat en el cas mitjà, però no el pitjor, en canvi, en la programació competitiva l'algoritme ha d'aconseguir un bon resultat fins i tot en el pitjor cas.

És per això que mai no construiríem una xarxa neuronal per resoldre un problema de programació competitiva, perquè aquest tipus de tècnica només s'aproxima a la resposta, en lloc de produir un resultat exacte.

Tot i que no entenc els complexos algorismes de les xarxes socials, la programació competitiva m'ha ajudat a entendre com la meva calculadora realitza ràpidament els seus càlculs! Visca!
\newline

\textbf{Finalment, podries fer una breu conclusió?}
\newline

La programació competitiva és divertida i realment augmenta el nostre pensament lògic i algorítmic, cosa que he trobat molt útil a la vida real fins i tot fora de les classes de matemàtiques o informàtica.

