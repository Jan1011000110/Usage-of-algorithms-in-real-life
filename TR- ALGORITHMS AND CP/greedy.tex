\subsection{Greedy}

El conegut i molt popular algorisme $Greedy$, és un algorisme que construeix una solució peça per peça, sempre escollint la peça que ens dona el benefici més evident i immediat. \newline
 
Imagineu que aneu a fer senderisme i que el vostre objectiu és arribar al cim més alt possible. Ja teniu el mapa abans de començar, però hi ha milers de camins possibles que es mostren al mapa. Ets massa mandrós i simplement no tens temps per avaluar-los. \newline

Comença a caminar amb una estratègia senzilla: agafar els camins que sembli que ens portaran més amunt, per tant, només cal agafar els camins que tenen més pendent. Sembla una bona estratègia per fer senderisme. Però és sempre el millor? \newline

Un cop acabat el viatge i tot el teu cos està adolorit i cansat, mires el mapa de senderisme per primera vegada. Oh Déu meu! Hi ha un riu fangós que hauria d'haver creuat, en lloc de continuar caminant cap amunt. Això vol dir que un algorisme $Greedy$ tria la millor opció immediata i mai reconsidera les seves eleccions. Pel que fa a l'optimització d'una solució, això simplement vol dir que la solució $Greedy$ intenta trobar solucions òptimes locals, que poden ser moltes, i es pot perdre una solució òptima global.\newline

Per això un algorisme $Greedy$ no és sempre efectiu, ja que encara que sempre triï el millor camí en cada decisió, potser, per agafar el millor camí total, primer havies d'agafar un camí aparentment dolent per després seguir amb camins més òptims i que portin a una solució global més favorable. \newline

Un exemple d'estratègia $Greedy$ a la vida real és la següent: imaginem que els nostres pares ens han donat 10 euros per comprar el que vulguem en una botiga de roba, com que volem aprofitar els diners que ens han donat, volem gastar el major nombre d'euros possibles, per tant, la nostra estratègia $Greedy$ podria consistir a mirar tots els productes ordenats de més cars a més barats i comprar cada producte si ens el podem permetre. \newline

Tot i així, aquesta estratègia és sempre òptima?

La realitat és que no, en alguns casos, per maximitzar la despesa, és millor comprar molts productes barats que no pas algun de car, per entendre-ho millor mirem els següents exemples: 


\begin{itemize}

\item Posem que els preus dels productes fossin els següents [2, 3, 3, 8] i que tenim un total de 10 euros, en aquest cas la nostra estratègia $Greedy$ funcionaria, ja que primer compraríem el producte de 8 euros i després el de 2 euros.

\item Posem que els preus dels productes fossin els següents [3, 3, 4, 9] i que tenim un total de 10 euros, en aquest cas la nostra estratègia $Greedy$ no funcionaria, puix que primerament compraríem el producte de 9 euros, en conseqüència ens quedaria 1 euro i no podríem comprar cap producte més, tot i així, en aquest cas, la compra òptima, seria obtenir els dos productes de 3 euros i el de 4 euros, en total ens gastaríem els 10 euros.

\end{itemize}

Seguidament, trobem la implementació de l'estratègia $Greedy$ comentada anteriorment.



\begin{lstlisting}
int nombreProductes; cin >> nombreProductes;

int dinersTotals; cin >> dinersTotals;

vector<int> preuProductes(nombreProductes);

for (int i = 0; i < nombreProductes; i++){
    cin >> preuProductes[i];
}

sort(preuProductes.rbegin(), preuProductes.rend());
// ordenem els preus de més car a més barat

int dinersGastats = 0;

for (int i = 0; i < nombreProductes; i++){
    if (dinersTotals >= preuProductes[i]){
        // si ens podem permetre el producte,
        // el comprem
        dinersGastats += preuProductes[i];
    }
}


cout << dinersTotals - dinersGastats << endl;
// dinersTotals - dinersGastats = 
// diners que ens han quedat 
\end{lstlisting}



